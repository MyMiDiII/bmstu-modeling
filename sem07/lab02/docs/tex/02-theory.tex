\chapter{Теоретическая часть}

\section{Марковские процессы}

Случаный процесс, протекающий в некоторой системе $S$, называется
\bfit{марковским}, если для каждого момента времени вероятность любого состояния
системы в будущем зависит только от ее состояния в настоящем времени и не
зависит от того, когда и каким образом система пришла в это состояние, то есть
не зависит от того, как процесс развивался в прошлом.

\section{Предельные вероятности состояний}

Для марковских процессов используются уравнения Колмогорова, составляющиеся по
следующему правилу:

\begin{enumerate}
    \item В левой части каждого уравнения стоит производная вероятности
        состояния.
    \item Правая часть чодержит столько членов, сколько стрелок связано с этим
        состоянием; если стрелка направлена из состояния соответствующий член
        имеет знак <<->>, если в состояние --- знак <<+>>.
    \item Каждый член равен плотности веротности перехода (интенсивности),
        соответсвующей данной стрелке, умноженной на вероятность того состояния,
        из которого исходит стрелка.
\end{enumerate}

То есть строится система уравнений, которые имеют вид:

\begin{equation}
    P_i'(t) = \sum\limits_{j=1}^{n} \lambda_{ji}P_j(t) - P_i(t)
    \sum\limits_{j=1}^{n} \lambda_{ij},
\end{equation}

где $P_i(t)$ -- вероятность того, что система находится в $i$-ом состоянии;

$n$ --- число состояний в системе;

$\lambda_ij$ --- интенсивность перехода системы из $i$-ого состояния в $j$-ое.

Одно из уравнений данной системы заменяется условием нормировки:

\begin{equation}
    \sum\limits_{i=1}^{n} P_i(t) = 1.
\end{equation}

В силу того, что \bfit{предельные вероятности состояний постоянны}, для их
определения в уравнениях Колмогорова необходимо \bfit{заменить их производные
нулями} и решить полученную систему линейных алгебраческих уравнений.

Отметим, что предельная вероятность состояния показывает \bfit{среднее
относительное время пребывания} системы в этом состоянии.

\section{Точки стабилизации}

Для определения точек стабилизации системы определяются вероятности состояний с
некоторым малым шагом $\Delta t$. Точка стабилизации считается найденной, если
приращение вероятности, а также разница между ранее найденной предельной
вероятностью состояния и вычисленной вероятностью, достаточно малы, то есть
выполняются соотношения:

\begin{equation}
    |P_i(t + \Delta t) -  P_i(t)| < \varepsilon,
\end{equation}

\begin{equation}
    |P_i(t) -  \lim_{t \rightarrow \infty} P_i(t)| < \varepsilon,
\end{equation}

где $\varepsilon$ --- заданная точность.
