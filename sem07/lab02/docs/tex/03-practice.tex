\chapter{Практическая часть}

\section{Текст программы}

На листинге~\ref{lst:prob} представлены функции расчета предельных вероятностей
по заданной матрице интенсивностей переходов.

{
\captionsetup{format=hang,justification=raggedright,
              singlelinecheck=off,width=16cm}
\listingfile{python}{probabilities.py}{}{Реализация функций расчета предельных
вероятностей}{prob}

На листинге~\ref{lst:stab} представлены функции расчета точек стабилизации.

\listingfile{python}{stabilization.py}{linerange=11-52}{Реализация функций
расчета точек стабилизации}{stab}
}

\clearpage
\section{Полученный результат}

На рисунках~\ref{img:table3} представлен пример работы программы для системы с
тремя состояниями. На рисунке~\ref{img:graph3} представлен соответствующий график
зависимости вероятности от времени.

\imgs{table3}{h!}{0.3}{Расчет для системы с тремя состояниями}

\imgw{graph3}{h!}{13cm}{График зависимости вероятности состояния от времени для
системы с тремя состояниями}

На рисунках~\ref{img:table5} представлен пример работы программы для системы с
пятью состояниями. На рисунке~\ref{img:graph5} представлен соответствующий график
зависимости вероятности от времени.

\imgs{table5}{h!}{0.4}{Расчет для системы с пятью состояниями}

\imgw{graph5}{h!}{17cm}{График зависимости вероятности состояния от времени для
системы с пятью состояниями}
