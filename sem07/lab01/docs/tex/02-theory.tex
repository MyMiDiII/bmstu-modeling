\chapter{Теоретическая часть}

\section{Равномерное распределение}

Случайная величина $X$ имеет \bfit{равномерное распределение} на
отрезке~$[a,~b]$, если ее плотность распределения~$f(x)$ равна:
\begin{equation}
    p(x) =
    \begin{cases}
        \displaystyle\frac{1}{b - a}, & \quad \text{если } a \leq x \leq b;\\
        0,  & \quad \text{иначе}.
    \end{cases}
\end{equation}

При этом функция распределения~$F(x)$ равна:

\begin{equation}
    F(x) =
    \begin{cases}
        0,  & \quad x < a;\\
        \displaystyle\frac{x - a}{b - a}, & \quad a \leq x \leq b;\\
        1,  & \quad x > b.
    \end{cases}
\end{equation}

Обозначение: $X \sim R[a, b]$.

\section{Нормальное распределение}

Случайная величина $X$ имеет \bfit{нормальное распределение} с
параметрами~$m$~и~$\sigma$, если ее плотность распределения~$f(x)$ равна:

\begin{equation}
    f(x) = \frac{1}{\sigma \cdot \sqrt{2\pi}}~~e^{\displaystyle-\frac{(x -
    m)^2}{2\sigma^2}}, \quad x \in \mathbb{R}, \sigma > 0.
\end{equation}

При этом функция распределения~$F(x)$ равна:

\begin{equation}
    F(x) = \frac{1}{\sigma \cdot \sqrt{2\pi}} \int\limits_{-\infty}^{x}
    e^{\displaystyle-\frac{(t - m)^2}{2\sigma^2}} dt,
\end{equation}

или, что то же самое:

\begin{equation}
    F(x) = \frac{1}{2} \cdot \bigg[1 + \erf\bigg(\frac{x - m}{\sigma
    \sqrt{2}}\bigg)\bigg],
\end{equation}

где $\erf(x) = $ \scalebox{1.3}{$\frac{2}{\sqrt{\pi}}$}
\scalebox{1.1}{$\int\limits_{0}^{x} e^{ -t^2} dt$} --- функция вероятности
ошибок.

Обозначение: $X \sim N(m, \sigma^2)$.
