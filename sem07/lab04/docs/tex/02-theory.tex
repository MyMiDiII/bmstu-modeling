\chapter{Теоретическая часть}

\section{Используемые распределения}

\subsection{Равномерное распределение}
Случайная величина $X$ имеет \bfit{равномерное распределение} на
отрезке~$[a,~b]$, если ее плотность распределения~$f(x)$ равна:
\begin{equation}
    p(x) =
    \begin{cases}
        \displaystyle\frac{1}{b - a}, & \quad \text{если } a \leq x \leq b;\\
        0,  & \quad \text{иначе}.
    \end{cases}
\end{equation}

Обозначение: $X \sim R[a, b]$.

Момент времени $t_i$ может быть вычислен по следующей формуле:

\begin{equation}
    t_i = a + (b - a) R, 
\end{equation}

где $R \in [0, 1]$ --- равномерно распределенная случайная величина в
промежутке~$[0, 1]$.

\subsection{Нормальное распределение}

Случайная величина $X$ имеет \bfit{нормальное распределение} с
параметрами~$m$~и~$\sigma$, если ее плотность распределения~$f(x)$ равна:

\begin{equation}
    f(x) = \frac{1}{\sigma \cdot \sqrt{2\pi}}~~e^{\displaystyle-\frac{(x -
    m)^2}{2\sigma^2}}, \quad x \in \mathbb{R}, \sigma > 0.
\end{equation}

Обозначение: $X \sim N(m, \sigma^2)$.

Момент времени $t_i$ может быть вычислен по следующей формуле:

\begin{equation}
    t_i = \sigma \sqrt\frac{12}{n}(\sum\limits_{i=1}^{n} R_i - \frac{n}{2}) + m,
\end{equation}

где $R_i \in [0, 1]$ --- равномерно распределенные случайные величины в
промежутке~$[0, 1]$.

\section{Описание принципов}

\subsection{Пошаговый принцип}

Пошаговый принцип или принцип $\Delta t$ заключается в последовательном анализе
состояний всех блоков в момент времени $t + \Delta t$ по заданному состоянию
блоков в момент времени $t$. При этом новое состояние блоков определяется в
соответствии с их алгоритмическим описанием с учетом действующих случайных
факторов. В результате этого анализа принимается решение о том, какие
общесистемные события должны имитироваться программой на данный момент времени.

Основной недостаток принципа $\Delta t$ заключается в значительных затратах
вычислительных ресурсов, а при недостаточно малом $\Delta t$ появляется
опасность пропуска отдельных событий в системе, исключающая возможность
получения правильных результатов при моделировании.

\subsection{Событийный принцип}

Состояния отдельных устройств изменяется в дискретные моменты времени,
совпадающие с моментами поступления сообщений в систему, окончания реализации
задания, поэтому моделирование и продвижение текущего времени в системе удобно
проводить, используя событийных принцип.

При использовании данного принципа состояние всех блоков имитационной модели
анализируется лишь в момент появления какого-либо события. Момент наступления
следующего события определяется минимальными значениями из списка будущих
событий, представляющего собой совокупность моментов ближайшего изменения
состояний каждого из блока системы.
