\chapter{Теоретическая часть}

\section{Методы получения последовательности случайных чисел}

Для генерации случайных чисел может применяться один из следующих способов:
\begin{itemize}
    \item \bfit{аппаратный}, в основе которого лежит какой-либо физический
        эффект (не реализуется в данной работе);
    \item \bfit{табличный}, при использовании которого заранее полученные и
        проверенные случайные числа оформлены в виде таблице в памяти ЭВМ;
    \item \bfit{алгоритмический}, с помощью которого формируются
        детерминированные последовательности чисел, где каждое число зависит от
        предыдущего, но для стороннего наблюдателя такие последовательности
        выглядят случайными, из-за чего называются псевдослучайными.
\end{itemize}

\subsection{Алгоритмический способ}

В данной работе реализуется \bfit{квадратичный когруэнтный метод}, в котором
последовательность чисел формируется следующим образом:

\begin{equation}
    y_{n+1} = (Ay_n^2 + By_n + C) \bmod m,
\end{equation}

где $m = 2^l$.

Если $l \geq 2$, то наибольшее значение периода квадратического конгруэнтного
датчика составляет $T_{\max} = 2^l$, что достигается при четном $A$, нечетном
$C$ и если нечетное $B$ удовлетворяется условию $B \bmod 4 = (A + 1) \bmod 4$.


\subsection{Табличный способ}

В данной работе для генерации случайных чисел табличным способом используются
цифры из части таблицы \bfit{<<A~Million Random Digits
with~100,000~Normal~Deviates>>}, опубликованной в 1955 году.

Данная таблица сохранена в виде текстового файла. Для генерации чисел выбирается
начальная позиция в файле, читаются следущие $n$ цифр, где $n$ --- количество
разрядов в генерируемом числе, и из стороковая последовательность преобразуется
в число. Для генерации следующиего числа происходит переход к следующей строке
таблице с сохранением номера столбца. При невозможности перейти к следующей
строке в связи с окончанием файла позиция переводится на первую строку, а номер
столбца увеличивается на единицу. Если цифр в строке не хватает для формирования
числа, они берутся из начала следующей строки.

\section{Критерий случайности}

Для оценки случайности был использован критерий на основе углов между векторами,
координаты начала и конца которых составляются из двух соседних пар
последовательности с одним общим числом.


