\chapter{Теоретическая часть}

\section{Методы получения последовательности случайных чисел}

Для генерации случайных чисел применяются следующие способы:
\begin{itemize}
    \item \bfit{аппаратный}, в основе которого лежат физические эффекты;
    \item \bfit{табличный}, при использовании которого заранее полученные и
        проверенные случайные числа оформлены в виде таблице в памяти ЭВМ;
    \item \bfit{алгоритмический}, с помощью которого формируются
        детерминированные последовательности чисел, где каждое число зависит от
        предыдущего, но для стороннего наблюдателя такие последовательности
        выглядят случайными, из-за чего называются псевдослучайными.
\end{itemize}

\subsection{Алгоритмический способ}

В данной работе реализуется \bfit{квадратичный когруэнтный метод}, в котором
последовательность чисел формируется следующим образом:

\begin{equation}
    y_{n+1} = (Ay_n^2 + By_n + C) \bmod m,
\end{equation}

где $m = 2^l$. Если $l \geq 2$, то наибольшее значение составляет $T_{\max} =
2^l$, что достигается при четном $A$, нечетном $C$ и если нечетное $B$
удовлетворяется условию $B \bmod 4 = (A + 1) \bmod 4$.


\subsection{Табличный способ}

В данной работе для генерации случайных чисел табличным способом используются
цифры из части таблицы \bfit{<<A~Million Random Digits
with~100,000~Normal~Deviates>>}, опубликованной в 1955 году.

Данная таблица сохранена в виде текстового файла. Для генерации чисел выбирается
начальная позиция в файле, читаются следущие $n$~цифр, где $n$ --- количество
разрядов в генерируемом числе, и из стороковая последовательность преобразуется
в число. Для генерации следующиего числа происходит переход к следующей строке
таблице с сохранением номера столбца. При невозможности перейти к следующей
строке в связи с окончанием файла позиция переводится на первую строку, а номер
столбца увеличивается на единицу. Если цифр в строке не хватает для формирования
числа, они берутся из начала следующей строки.

\section{Критерий случайности}

В данной работе для оценки случайности предлагается критерий автора на основе
углов между векторами, координаты начала и конца которых составляются из двух
соседних пар последовательности с одним общим числом. Критерий состоит в
следующем.

Из элементов сгенерированной последовательности $x_1, x_2, ..., x_N$ длиной~$N$
формируются пары $(x_i, x_{i+1})$, где $i = 1...N-1$. Далее каждая из этой пары
воспринимается, как координаты точки на плоскости.

Для каждой пары координат точек $(x_i, x_{i+1}), (x_{i + 1}, x_{i+2})$ ищется их
соединяющий вектор $\vec{v_i}$ с координатами $(x_{i + 1} - x_i, x_{i + 2} -
x_{i + 1})$. Из векторов также формируется последовательность.

Далее для каждого вектора происходит поиск углов между ним и $k$ ближайших
векторов в последовательности. Угол между векторами $\theta$ определяется по
формуле:

\begin{equation}
    \cos{\theta} = \frac{\vec{v_i} \cdot \vec{v_j}}{\|\vec{v_i}\|\|\vec{v_j}\|}.
\end{equation}

Углы равные $0$ и $\pi$ являются признаком линейной зависимости, поэтому им
сопоставляется коэффициент $K = 0$. При данном алгоритме формирования координат
точек из последовательность угол величиной $\frac{\pi}{2}$ между векторами,
параллельными осям координат свидетельсвует о повторении чисел, однако
повторение чисел в случайной последовательности приемлемо, при этом угол
величиной $\frac{\pi}{2}$ может быть получен и при других положениях векторов,
поэтому данному углу сопоставляется коэффициент равные $K = \frac{1}{2}$. Углам
величиной $\frac{\pi}{4}$ и $\frac{3\pi}{4}$ сопоставляется коэффициент $K = 1$,
как средним между названными выше углами. Коэффиценты $K$ для остальных углов
рассчитываются по формуле ($\theta' = \theta$, если $\theta \leq \frac{\pi}{2}$,
$\theta' = \pi - \theta$, иначе):

\begin{equation}
    K = \begin{cases}
        \displaystyle\frac{4\theta'}{\pi}, & \quad \text{если } 0 \leq
        \theta' < \frac{\pi}{4};\\
        \displaystyle\frac{-2\theta'}{\pi} +
        \displaystyle\frac{3}{2},  & \quad \text{если } \frac{\pi}{4} \leq
        \theta' \leq \frac{\pi}{2},
    \end{cases}
\end{equation}

Итоговый коэффициент лежащий в промежутке $[0, 1]$ считается как отношение суммы
найденный коэффициентов $K_i$ к количеству найденных углов.
