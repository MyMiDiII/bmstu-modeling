\chapter{Практическая часть}

\section{Текст программы}

{
\captionsetup{format=hang,justification=raggedright,
              singlelinecheck=off,width=16cm}
На листинге~\ref{lst:alg} представлена реализация квадратичного
когруэнтного метода генерации случайных чисел.

\listingfile{python}{algorithmic.py}{}{Реализация квадратичного когруэнтного
метода генерации случайных чисел}{alg}

\clearpage
На листинге~\ref{lst:tab} представлена реализация табличного способа
получения последовательности случайных чисел.

\listingfile{python}{tabular.py}{}{Реализация табличного способа генерации
случайных чисел}{tab}

\clearpage
На листинге~\ref{lst:crit} представлена функция вычисления коэффициента
описанного выше критерия оценки случайности.

\listingfile{python}{criterion.py}{}{Реализация функции расчета коэффициента
критерия оценки случайности}{crit}
}

\section{Полученный результат}

На рисунке~\ref{img:algtab} приведен результат работы программы при генерации
5000~чисел алгоритмическим и табличным методами.

На рисунке~\ref{img:const} приведен результат расчета коэффициента критерия
случайности для постоянной последовательности.

На рисунке~\ref{img:up} приведен результат расчета коэффициента критерия
случайности для строго возрастающей последовательности.

На рисунке~\ref{img:down} приведен результат расчета коэффициента критерия
случайности для строго убывающей последовательности.

На рисунке~\ref{img:period} приведен результат расчета коэффициента критерия
случайности для периодической последовательности.

На рисунке~\ref{img:rand10} приведен результат расчета коэффициента критерия
случайности для случайной последовательности из 10 чисел.
