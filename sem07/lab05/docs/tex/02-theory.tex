\chapter{Теоретическая часть}

\section{Схемы модели}

На рисунке~\ref{img:struct} представлена структурная схема модели.

\imgs{struct}{h!}{1}{Структурная схема модели}

В процессе взаимодействия клиентов с информационным центром возможно два режима
работы:

\begin{itemize}
    \item режим нормального обслуживания, когда клиент выбирает одного из
        свободных операторов, отдавая предпочтение тому, у кого максимальная
        производительность;
    \item режим отказа клиенту в обслуживании, когда все операторы заняты.
\end{itemize}

На рисунке~\ref{img:mss} представлена схема модели в терминах систем массового
обслуживания~(СМО).

\vspace{0.5cm}
\imgs{mss}{h!}{1.2}{Схема модели в терминах СМО}

\section{Переменные и уравнение имитационной модели}

Эндогенные переменные:
\begin{itemize}
    \item время обработки задания $i$-ым оператором;
    \item время решения задания на $j$-ом компьютере.
\end{itemize}

Экзогенные переменные:
\begin{itemize}
    \item $n_0$ --- число обслуженных клиентов;
    \item $n_1$ --- число клиентов, получивших отказ.
\end{itemize}

Вероятность отказа рассчитывается по формуле~\ref{eq:01}, которая описывает
уравнение модели:

\begin{equation}\label{eq:01}
    P_{\text{отказа}} = \frac{n_0}{n_0 + n_1}.
\end{equation}
