\chapter{Теоретическая часть}

\section{Используемые распределения}

\subsection{Равномерное распределение}
Случайная величина $X$ имеет \bfit{равномерное распределение} на
отрезке~$[a,~b]$, если ее плотность распределения~$f(x)$ равна:
\begin{equation}
    p(x) =
    \begin{cases}
        \displaystyle\frac{1}{b - a}, & \quad \text{если } a \leq x \leq b;\\
        0,  & \quad \text{иначе}.
    \end{cases}
\end{equation}

Обозначение: $X \sim R[a, b]$.


\subsection{Нормальное распределение}

Случайная величина $X$ имеет \bfit{нормальное распределение} с
параметрами~$m$~и~$\sigma$, если ее плотность распределения~$f(x)$ равна:

\begin{equation}
    f(x) = \frac{1}{\sigma \cdot \sqrt{2\pi}}~~e^{\displaystyle-\frac{(x -
    m)^2}{2\sigma^2}}, \quad x \in \mathbb{R}, \sigma > 0.
\end{equation}

Обозначение: $X \sim N(m, \sigma^2)$.

\section{Схема модели}

На рисунке~\ref{img:scheme} представлена схема модели в терминах систем
массового облуживания~(СМО).

\imgs{scheme}{h!}{1.5}{Схема модели в терминах СМО}
