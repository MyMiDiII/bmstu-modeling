\textbf{Тема:} Программная реализация приближенного аналитического метода и
численных алгоритмов первого и второго порядков точности при решении задачи Коши
для ОДУ.

\textbf{Цель работы.} Программная реализация приближенного аналитического метода
и численных алгоритмов первого и второго порядков точности при решении задачи
Коши для ОДУ.


\spchapter{Исходные данные}

1. ОДУ, не имеющее аналитического решения (формула~(\ref{eq:01})).

\begin{equation}\label{eq:01}
    \begin{aligned}
        u'(x) &= x^2 + u^2,\\
        u(0)~ &= 0.
    \end{aligned}
\end{equation}

\spchapter{Описание алгоритмов}

Обыкновенные дифференциальные уравнения~(ОДУ)~--- дифференциальные
уравнения~(ДУ) с одной независимой переменной.

ДУ $n$-ого порядка описывается формулой~(\ref{eq:02}). Заменой
переменной ОДУ $n$-ого порядка сводится к системе ДУ первого порядка.

\begin{equation}\label{eq:02}
    F(x, u, u', u'', ..., u^{(n)}) = 0.
\end{equation}

Задача данной лабораторной работы является задачей Коши, состоящей в поиске
решения дифференциального уравнения, удовлетворяющего начальным условия
(формула~(\ref{eq:03})).

\begin{equation}\label{eq:03}
    \begin{aligned}
        u'(x) &= f(x, u),\\
        u(\xi)~ &= \eta.
    \end{aligned}
\end{equation}

В данной лабораторной работе рассматриваются следующие методы решения:
\begin{itemize}
    \item метод Пикара;
    \item явный метод первого порядка точности (Эйлера);
    \item явный метод второго порядка точность (Рунге-Кутта).
\end{itemize}

\section{Метод Пикара}

Пусть поставлена задача Коши, выражющаяся формулой~(\ref{eq:04}):

\begin{equation}\label{eq:04}
    \begin{aligned}
        u'(x) &= \varphi(x, u(x)),\\
        x_0 \leq &~x \leq x_l\\
        u(x_0) &= u_0.
    \end{aligned}
\end{equation}

Проитегрировав выписанное уравнение получим формулу~(\ref{eq:05}).

\begin{equation}\label{eq:05}
    u(x) = u_0 + \int\limits_{x_0}^{x}\varphi(t, u(t))dt.
\end{equation}

Последовательные приближения метода пикара реализуются по схеме,
описывающейся формулой~(\ref{eq:06}).

\begin{equation}\label{eq:06}
    y_{i}(x) = y_0 + \int\limits_{x_0}^{x}\varphi(t, y_{i - 1}(t))dt,
\end{equation}

где $i = 1, 2, ...$ --- номер итерации,

причем $u_0(t) = y_0$.

Для задачи данной лабораторной работы с помощью схемы, описывающейся
формулой~(\ref{eq:06}), получим следующие приближения
(формулы~(\ref{eq:07}-\ref{eq:10})):

\begin{equation}\label{eq:07}
    y_1(x) = 0 + \int\limits_0^x(t^2 + y_0^2(t))dt = \int\limits_0^x t^2 dt =
    \frac{t^3}{3}\bigg|_0^x = \frac{x^3}{3},
\end{equation}

\begin{equation}\label{eq:08}
    \begin{split}
    y_2(x) = 0 + \int\limits_0^x(t^2 + y_1^2(t))dt =
    \int\limits_0^x\bigg(t^2 + \bigg(\frac{t^3}{3}\bigg)^2\bigg) dt = \\
    = \int\limits_0^x\bigg(t^2 + \frac{t^6}{9}\bigg) dt =
    \bigg(\frac{t^3}{3} + \frac{t^7}{63}\bigg)\bigg|_0^x =
    \frac{x^3}{3} + \frac{x^7}{63}
    \end{split}
\end{equation}

\begin{equation}\label{eq:09}
    \begin{split}
        y_3(x) &= 0 + \int\limits_0^x(t^2 + y_2^2(t))dt = \\ &=
        \int\limits_0^x\bigg(t^2 + \bigg(\frac{t^3}{3} +
        \frac{t^7}{63}\bigg)^2\bigg) dt = \int\limits_0^x\bigg(t^2 +
        \frac{t^6}{9} + \frac{2t^{10}}{63 \cdot 3} + \frac{t^{14}}{63^2}\bigg) dt =
        \\ &= \bigg(\frac{t^3}{3} + \frac{t^7}{63} + \frac{2t^{11}}{2079} +
        \frac{t^{15}}{59535}\bigg)\bigg|_0^x = \frac{x^3}{3} + \frac{x^7}{63} +
        \frac{2x^{11}}{2079} + \frac{x^{15}}{59535}
    \end{split}
\end{equation}

\begin{equation}\label{eq:10}
    \begin{split}
        y_4(x) &= 0 + \int\limits_0^x(t^2 + y_3^2(t))dt = \\ &=
        \int\limits_0^x\bigg(t^2 + \bigg(\frac{x^3}{3} + \frac{x^7}{63} +
        \frac{2x^{11}}{2079} + \frac{x^{15}}{59535}\bigg)^2\bigg) dt = \\ &=
        \bigg(\frac{t^3}{3} + \frac{t^7}{63} + \frac{2t^{11}}{2079} +
        \frac{13t^{15}}{218295} +  \frac{82t^{19}}{37328445} +
        \frac{662t^{23}}{10438212015} + \\ & + \frac{4t^{27}}{3341878155} +
        \frac{t^{31}}{109876901975}\bigg)\bigg|_0^x =
        \frac{x^3}{3} + \frac{x^7}{63} + \frac{2x^{11}}{2079} +
        \frac{13x^{15}}{218295} + \\ &+ \frac{82x^{19}}{37328445} +
        \frac{662x^{23}}{10438212015} + \frac{4x^{27}}{3341878155} +
        \frac{x^{31}}{109876902975}
    \end{split}
\end{equation}

\section{Метод Эйлера}

По методу Эйлера первого порядка точности значение функции в точке
вычисляется по формуле~(\ref{eq:11}).

\begin{equation}\label{eq:11}
    y_{n+1} = y_n + h \cdot f(x_n, y_n).
\end{equation}

\section{Метод Рунге-Кутта}

По методу Рунге-Кутта второго порядка точности значение функции в точке
вычисляется по формуле~(\ref{eq:12}).

\begin{equation}\label{eq:12}
    y_{n+1} = y_n + h((1 - \alpha) F_1 + \alpha F_2),
\end{equation}

где $F_1 = f(x_n, y_n)$,

~~~~~$F_2 = f(x_n + \frac{h}{2\alpha}, y_n + \frac{h}{2\alpha}F_1)$,

~~~~~$\alpha = \frac{1}{2}$ или $\alpha = 1$

\spchapter{Код программы}

На листинге \ref{lst:picard} представлена реализация метода Пикара.
На листиге \ref{lst:func} представлено вычисление правой части уравнения. На
листинге \ref{lst:euler} представлена реализация метода Эйлера.
На листинге \ref{lst:rungekutta} представлена реализация метода Рунге-Кутта.

\mybreaklisting{main.py}{11-38}{Реализация метода Пикара}{picard}

\mybreaklisting{main.py}{41-42}{Вычисление правой части уравнения}{func}

\mybreaklisting{main.py}{45-54}{Реализация метода Эйлера}{euler}

\mybreaklisting{main.py}{56-70}{Реализация метода Рунге-Кутта}{rungekutta}

\spchapter{Результат работы}

На рисунке \ref{img:table} представлена таблица результов вычисления
значений функции, являющейся решением ОДУ. На рисунках
\ref{img:picard}-\ref{img:rungekutta}
представлены графики функции-решения, вычисленной различными методами.

\img{14cm}{table}{Полученная таблица значений}{table}
\img{8.5cm}{picard}{График функции, вычисленной методом Пикара}{picard}
\img{5cm}{euler}{График функции, вычисленной методом Эйлера}{euler}
\img{5cm}{rungekutta}{График функции, вычисленной методом
Рунге-Кутта}{rungekutta}

\spchapter{Контрольные вопросы}

\begin{enumerate}[label=\textbf{\arabic*})]
    \item \textbf{Укажите интервалы значений аргумента, в которых можно считать решением
        заданного уравнения каждое из первых 4-х приближений Пикара, т.~е. для
        КАЖДОГО приближения указать свои границы применимости. Точность
        результата оценивать до второй цифры после запятой. Объяснить свой
        ответ.}

        Интервалы значений аргумента, в которых можно считать решением
        заданного уравнения каждого из приближений Пикара, можно найти из
        полученной таблицы решений (рисунок~\ref{img:table}), а
        именно, $i$-ое приближение Пикара ($i = \overline{1, 4}$) можно считать
        решением до тех пор, пока его значение совпадает с $(i+1)$-ым
        приближением Пикара с некоторой точностью (по условию до второй
        цифры после запятой). Таким образом:

        \begin{itemize}
            \item 1-ое приближение Пикара можно считать решением на
                промежутке $[0, 0.85]$;
            \item 2-ое приближение --- на промежутке $[0, 1]$;
            \item 3-е приближение --- на промежутке $[0, 1.30]$;
            \item для определения промежутка для 4-ого приближения Пикара
                необходимо найти 5-ое приближение, чего в данной работе не
                проводилось.
        \end{itemize}

    \item \textbf{Пояснить, каким образом можно доказать правильность полученного
        результата при фиксированном значении аргумента в численных методах.}

        В численных методах для того, чтобы доказать правильность полученного
        результата при фиксированном значении аргумента, необходимо итерационно
        уменьшать шаг до нуля, вычисляя для каждого шага значение фукнции, до
        тех пор пока, уменьшение шага не перестанет приводить к изменению
        решения с некоторой точностью.

    \item \textbf{Каково значение решения уравнения в точке $x=2$, т.~е. привести значение
        $u(2)$.}

        $\approx 318$

    \item \textbf{Дайте оценку точки разрыва решения уравнения.}

        ---


    \item \textbf{Покажите, что метод Пикара сходится к точному аналитическому решению
        уравнения:}

        \begin{equation*}
            \begin{aligned}
                u'(x) &= x^2 + u,\\
                u(0)~ &= 0.
            \end{aligned}
        \end{equation*}

        Найдем аналитическое решение методом Бернулли. Пусть:

        \begin{equation*}
            \begin{aligned}
                u &= ab,\\
                a &= a(x),\\
                b &= b(x).
            \end{aligned}
        \end{equation*}

        Тогда:

        \begin{equation*}
            \begin{aligned}
                u' &= a'b + b'a.\\
            \end{aligned}
        \end{equation*}

        Откуда:

        \begin{equation*}
            \begin{aligned}
                a'b &+ b'a = x^2 + ab,\\
                b(a' &- a) + b'a = x^2.
            \end{aligned}
        \end{equation*}

        Т.~к. $a$ можно выбрать произвольно:

        \begin{equation*}
            \begin{cases}
                a' - a &=0,\\
                b'a &= x^2.
            \end{cases}
        \end{equation*}

        Найдем $a$:

        \begin{equation*}
            \begin{aligned}
                a' - a &= 0,\\
                \frac{da}{dx} &= a,\\
                \frac{da}{a} &= dx,\\
                \int\frac{da}{a} &= \int dx,\\
                \log a &= x,\\
                e^{\log a} &= e^x,\\
                a &= e^x.
            \end{aligned}
        \end{equation*}

        Подставим $a$ во второе уравнение системы и найдем $b$:

        \begin{equation*}
            \begin{aligned}
                b'e^x &= x^2,\\
                \frac{db}{dx}e^x &= x^2,\\
                db = x^2 & e^{-x} dx,\\
                \int db = \int x^2 & e^{-x} dx,\\
                b = \int x^2 & e^{-x} dx.
            \end{aligned}
        \end{equation*}

        Дважды интегрируем по частям:

        \begin{equation*}
            \begin{split}
                b = -x^2 e^{-x} + 2\int xe^{-x} dx = -x^2e^{-x} -2xe^{-x} +
                2\int e^{-x} dx = \\ = -x^2 e^{-x} - 2xe^{-x} - 2e^{-x} + C =
                -e^{-x}(x^2 + 2x + 2) + C.
            \end{split}
        \end{equation*}

        Тогда:
        \begin{equation*}
            \begin{split}
                u = ab = e^x(-e^{-x}(x^2 + 2x + 2) + C) = -(x^2 + 2x + 2) + Ce^x.
            \end{split}
        \end{equation*}

        По начальному условию:
        \begin{equation*}
            \begin{aligned}
                0 &= -2 + Ce^0,\\
                C &= 2.
            \end{aligned}
        \end{equation*}

        Таким образом, аналитическое решение:
        \begin{equation}\label{eq:13}
            u = 2e^x - x^2 - 2x - 2.
        \end{equation}

        Приближения Пикара:
        \begin{equation*}
            y_1(x) = 0 + \int\limits_0^x(t^2 + y_0(t))dt = \int\limits_0^x t^2
            dt = \frac{x^3}{3},
        \end{equation*}

        \begin{equation*}
            y_2(x) = 0 + \int\limits_0^x(t^2 + y_1(t))dt = \int\limits_0^x
            \bigg(t^2 + \frac{t^3}{3}\bigg)dt = \frac{x^3}{3} + \frac{x^4}{12},
        \end{equation*}

        \begin{equation*}
            y_3(x) = 0 + \int\limits_0^x(t^2 + y_2(t))dt = \int\limits_0^x
            \bigg(t^2 + \frac{t^3}{3} + \frac{x^4}{12}\bigg)dt = \frac{x^3}{3} +
            \frac{x^4}{12} + \frac{x^5}{60},
        \end{equation*}
        
        \begin{equation*}
            ...
        \end{equation*}

        Для $n$-ого приближения:
        \begin{equation*}
            y_n(x) = 0 + \int\limits_0^x(t^2 + y_{n-1}(t))dt =
            2\bigg(\frac{x^3}{3!}
            + \frac{x^4}{4!} + \frac{x^5}{5!} + ... + \frac{x^n}{n!}\bigg).
        \end{equation*}

        Устремим $n$ к бесконечности:
        \begin{equation*}
            \begin{split}
                y(x) = \lim_{n \to \infty} y_n(x) = \lim_{n \to \infty}
                2\bigg(\frac{x^3}{3!} + \frac{x^4}{4!} + \frac{x^5}{5!} + ... +
                \frac{x^n}{n!}\bigg) = \\ = 2 \sum\limits_{n=3}^{\infty}
                \frac{x^n}{n!} = 2\bigg( \sum\limits_{n=0}^{\infty} \frac{x^n}{n!} -
                1 - x - \frac{x^2}{2}\bigg).
            \end{split}
        \end{equation*}

        При этом по формуле Маклорена:
        \begin{equation*}
            \sum\limits_{n=0}^{\infty} \frac{x^n}{n!} = e^x
        \end{equation*}

        С учетом этого:
        \begin{equation}\label{eq:14}
            y(x) = 2e^x - x^2 - 2x - 2.
        \end{equation}

        Таким образом, метод Пикара (формула~(\ref{eq:14})) сходится к
        аналитическому решинию (формула~(\ref{eq:13})).
\end{enumerate}
