\textbf{Тема:} Программная реализация приближенного аналитического метода и
численных алгоритмов первого и второго порядков точности при решении задачи Коши
для ОДУ.

\textbf{Цель работы.} Программная реализация приближенного аналитического метода
и численных алгоритмов первого и второго порядков точности при решении задачи
Коши для ОДУ.


\spchapter{Исходные данные}

1. ОДУ, не имеющее аналитического решения (формула~(\ref{eq:01})).

\begin{equation}\label{eq:01}
    \begin{aligned}
        u'(x) &= x^2 + u^2,\\
        u(0)~ &= 0.
    \end{aligned}
\end{equation}

\spchapter{Описание алгоритмов}

Обыкновенные дифференциальные уравнения~(ОДУ)~--- дифференциальные
уравнения~(ДУ) с одной независимой переменной.

ДУ $n$-ого порядка описывается формулой~(\ref{eq:02}). Заменой
переменной ОДУ $n$-ого порядка сводится к системе ДУ первого порядка.

\begin{equation}\label{eq:02}
    F(x, u, u', u'', ..., u^{(n)}) = 0.
\end{equation}

Задача данной лабораторной работы является задачей Коши, состоящей в поиске решения
дифференциального уравнения, удовлетворяющего начальным условия (формула~(\ref{eq:02})).

\begin{equation}\label{eq:02}
    \begin{aligned}
        u'(x) &= f(x, u),\\
        u(\xi)~ &= \eta.
    \end{aligned}
\end{equation}

В данной лабораторной работе рассматриваются следующие методы решения:
\begin{itemize}
    \item метод Пикара;
    \item явный метод первого порядка точности (Эйлера);
    \item явный метод второго порядка точность (Рунге-Кутта).
\end{itemize}

\section{Метод Пикара}

\spchapter{Код программы}

\spchapter{Результат работы}

\spchapter{Контрольные вопросы}
