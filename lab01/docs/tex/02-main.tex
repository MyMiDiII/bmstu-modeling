\textbf{Тема:} Программная реализация приближенного аналитического метода и
численных алгоритмов первого и второго порядков точности при решении задачи Коши
для ОДУ.

\textbf{Цель работы.} Программная реализация приближенного аналитического метода
и численных алгоритмов первого и второго порядков точности при решении задачи
Коши для ОДУ.


\spchapter{Исходные данные}

1. ОДУ, не имеющее аналитического решения (формула~(\ref{eq:01})).

\begin{equation}\label{eq:01}
    \begin{aligned}
        u'(x) &= x^2 + u^2,\\
        u(0)~ &= 0.
    \end{aligned}
\end{equation}

\spchapter{Описание алгоритмов}

Обыкновенные дифференциальные уравнения~(ОДУ)~--- дифференциальные
уравнения~(ДУ) с одной независимой переменной.

ДУ $n$-ого порядка описывается формулой~(\ref{eq:02}). Заменой
переменной ОДУ $n$-ого порядка сводится к системе ДУ первого порядка.

\begin{equation}\label{eq:02}
    F(x, u, u', u'', ..., u^{(n)}) = 0.
\end{equation}

Задача данной лабораторной работы является задачей Коши, состоящей в поиске
решения дифференциального уравнения, удовлетворяющего начальным условия
(формула~(\ref{eq:03})).

\begin{equation}\label{eq:03}
    \begin{aligned}
        u'(x) &= f(x, u),\\
        u(\xi)~ &= \eta.
    \end{aligned}
\end{equation}

В данной лабораторной работе рассматриваются следующие методы решения:
\begin{itemize}
    \item метод Пикара;
    \item явный метод первого порядка точности (Эйлера);
    \item явный метод второго порядка точность (Рунге-Кутта).
\end{itemize}

\section{Метод Пикара}

Пусть поставлена задача Коши, выражющаяся формулой~(\ref{eq:04}):

\begin{equation}\label{eq:04}
    \begin{aligned}
        u'(x) &= \varphi(x, u(x)),\\
        x_0 \leq &~x \leq x_l\\
        u(x_0) &= u_0.
    \end{aligned}
\end{equation}

Проитегрировав выписанное уравнение получим формулу~(\ref{eq:05}).

\begin{equation}\label{eq:05}
    u(x) = u_0 + \int\limits_{x_0}^{x}\varphi(t, u(t))dt.
\end{equation}

Последовательные приближения метода пикара реализуются по схеме,
описывающейся формулой~(\ref{eq:06}).

\begin{equation}\label{eq:06}
    u_{i}(x) = u_0 + \int\limits_{x_0}^{x}\varphi(t, u_{i - 1}(t))dt,
\end{equation}

где $i = 1, 2, ...$ --- номер итерации,

причем $u_0(t) = u_0$.

Для задачи данной лабораторной работы с помощью схемы, описывающейся
формулой~(\ref{eq:06}), получим следующие приближения
(формулы~(\ref{eq:07}-\ref{eq:10})):

\begin{equation}\label{eq:07}
    u_1(x) = 0 + \int\limits_0^x(t^2 + u_0^2(t))dt = \int\limits_0^x t^2 dt =
    \frac{t^3}{3}\bigg|_0^x = \frac{x^3}{3},
\end{equation}

\begin{equation}\label{eq:08}
    \begin{split}
    u_2(x) = 0 + \int\limits_0^x(t^2 + u_1^2(t))dt =
    \int\limits_0^x\bigg(t^2 + \bigg(\frac{t^3}{3}\bigg)^2\bigg) dt = \\
    = \int\limits_0^x\bigg(t^2 + \frac{t^6}{9}\bigg) dt =
    \bigg(\frac{t^3}{3} + \frac{t^7}{63}\bigg)\bigg|_0^x =
    \frac{x^3}{3} + \frac{x^7}{63}
    \end{split}
\end{equation}

\begin{equation}\label{eq:09}
    \begin{split}
        u_3(x) &= 0 + \int\limits_0^x(t^2 + u_2^2(t))dt = \\ &=
        \int\limits_0^x\bigg(t^2 + \bigg(\frac{t^3}{3} +
        \frac{t^7}{63}\bigg)^2\bigg) dt = \int\limits_0^x\bigg(t^2 +
        \frac{t^6}{9} + \frac{2t^{10}}{63 \cdot 3} + \frac{t^{14}}{63^2}\bigg) dt =
        \\ &= \bigg(\frac{t^3}{3} + \frac{t^7}{63} + \frac{2t^{11}}{2079} +
        \frac{t^{15}}{59535}\bigg)\bigg|_0^x = \frac{x^3}{3} + \frac{x^7}{63} +
        \frac{2x^{11}}{2079} + \frac{x^{15}}{59535}
    \end{split}
\end{equation}

\begin{equation}\label{eq:10}
    \begin{split}
        u_4(x) &= 0 + \int\limits_0^x(t^2 + u_3^2(t))dt = \\ &=
        \int\limits_0^x\bigg(t^2 + \bigg(\frac{x^3}{3} + \frac{x^7}{63} +
        \frac{2x^{11}}{2079} + \frac{x^{15}}{59535}\bigg)^2\bigg) dt = \\ &=
        \bigg(\frac{t^3}{3} + \frac{t^7}{63} + \frac{2t^{11}}{2079} +
        \frac{13t^{15}}{218295} +  \frac{82t^{19}}{37328445} +
        \frac{662t^{23}}{10438212015} + \\ & + \frac{4t^{27}}{3341878155} +
        \frac{t^{31}}{109876901975}\bigg)\bigg|_0^x =
        \frac{x^3}{3} + \frac{x^7}{63} + \frac{2x^{11}}{2079} +
        \frac{13x^{15}}{218295} + \\ &+ \frac{82x^{19}}{37328445} +
        \frac{662x^{23}}{10438212015} + \frac{4x^{27}}{3341878155} +
        \frac{x^{31}}{109876902975}
    \end{split}
\end{equation}

\spchapter{Код программы}

\spchapter{Результат работы}

\spchapter{Контрольные вопросы}
